\section{Research Methods}
\subsection{Research methodology}
%Research methodology description
The research methodology employed for the purpose of this research will be purely Quantitative. The design parameters for the implemented model are discrete, e.g. number of probes sent per hop. The measured data will be both discrete and continuous. The implemented software will be written in the Go language.


%Expand on parameters
\subsection{Research methods}
%Research methods description
The precision, recall, and numbers
of probes sent using the improved dublin traceroute will be used as the basis of establishing model performance. They will be evaluated using ContainerLab \cite{containerlab} which is an open source network emulator which allows the creation of virtual network environments, it supports containerized router images of products offers by major companies such as Cicso, Juniper and Dell. This avoids ethical implications of evaluating performance on the real-world internet. 


\subsection{Data analysis}
%Data analysis description
Analysis of the implemented model will be carried out using several metrics in order to illustrate comparisons between it's performance and the original model's performance. Several plot types such as scatter plots, line plots and violin plots will be used to visualize; negative and positive correlations and also to represent the standard deviation of results respectively. Furthermore, tables of data will also be used to provide accurate numerical data in a straightforward manner.